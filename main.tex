\documentclass[11pt]{article} % use larger type; default would be 10pt
\usepackage[utf8]{inputenc} % set input encoding (not needed with XeLaTeX)
\usepackage{geometry} % to change the page dimensions
\geometry{a4paper} % or letterpaper (US) or a5paper or....
\usepackage{graphicx} % support the \includegraphics command and options
%%% PACKAGES
\usepackage{booktabs} % for much better looking tables
\usepackage{array} % for better arrays (eg matrices) in maths
\usepackage{paralist} % very flexible & customisable lists (eg. enumerate/itemize, etc.)
\usepackage{verbatim} % adds environment for commenting out blocks of text & for better verbatim
\usepackage{subfig} % make it possible to include more than one captioned figure/table in a single float
\usepackage{fancyhdr} % This should be set AFTER setting up the page geometry
\pagestyle{fancy} % options: empty , plain , fancy
\renewcommand{\headrulewidth}{0pt} % customise the layout...			
\lhead{}\chead{}\rhead{}
\lfoot{}\cfoot{\thepage}\rfoot{}

\usepackage{sectsty}
\allsectionsfont{\sffamily\mdseries\upshape} % (See the fntguide.pdf for font help)
% (This matches ConTeXt defaults)

\usepackage[nottoc,notlof,notlot]{tocbibind} % Put the bibliography in the ToC
\usepackage[titles,subfigure]{tocloft} % Alter the style of the Table of Contents
\renewcommand{\cftsecfont}{\rmfamily\mdseries\upshape}
\renewcommand{\cftsecpagefont}{\rmfamily\mdseries\upshape} % No bold!

\title{Movie Reviews}
\author{}
%\date{} % Activate to display a given date or no date (if empty),
         % otherwise the current date is printed 

\begin{document}
\maketitle

\section{3 Idiots}
\subsection{Summary}
Two friends from college set out on a journey to find their third friend. On this journey, they encounter a long forgotten bet, a wedding they must crash, and a funeral that goes impossibly out of control. As they make their way through the perilous landscape, another journey begins: their inner journey through memory lane and the story of their friend--the irrepressible free-thinker Rancho, who in his unique way, touched and changed their lives. It’s a story of their hostel days that swings between Rancho’s romance with the spirited Pia, and his clash with an oppressive mentor, Viru Sahastrabudhhe . And then one day, suddenly, Rancho is gone.

\subsection{Review}
Made a change here.\\
changed again
3 Idiots is a coming-age comedy and inspirational film directed by Rajkumar Hirani. The story of the film is similar to that of the famous novel Five Point Someone by Chetan Bhagat. This film is based on our Indian educational system which pressurizes the students and has an obsession for high grades.  Aamir Khan as Rancho plays the role of a very bright student who motivates his friends pursue their passion and make a career out of it. R.Madhavan as Farhan who wants to become a wildlife photographer but was forced to pursue engineering because of his strict father. Sharman Joshi as Raju who is pursuing engineering with the goal of getting his family out of extreme poverty. Boman Irani as 'ViruS' who plays the role of a strict college director who is highly competitive. Kareena Kapoor as Pia who plays the role of virus’s daughter and falls in love with Aamir khan.

The film goes very smooth. The film’s first half includes scenes such as ragging, confrontations with the professor, Chatur’s speech and mournful meal at Raju’s home. Overall the film depicts a very nice mingle of comedy, suspense and motivational. All the actors and co-actors done a commendable job. The music of the film was very soothing. The back ground music supports the film all through the movie. Overall depiction is fine. Some of the scenes are really emotional. Overall rating is excellent.Its definitely a one time watch.


But the movie has only average cinematography. It has been shot at some places including manali and Mussorie, which further add value to the movie.

\subsection{Starring}
\begin{itemize}
	\item Aamir Khan - Rancho
	\item R. Madhavan - Farhan Qureshi
	\item Sharman Joshi - Raju Rastogi
	\item Kareena Kapoor - Pia Sahastrabuddhe
	\item Boman Irani - Viru Sahastrabuddhe
\end{itemize}

\section{Chak de! India}
\subsection{Review}
Chak De! India is the basic, every-single-sports-movie story of a disgraced player, here called Kabir Khan(Shahrukh Khan), pulling together a team of misfits to do the impossible -- here winning the World Championship. 

We start, of course, with the fall. Kabir, India's most successful Centre Forward of all time, flubs a crucial penalty and is castigated by his nation -- an Islamic last name and a meteoric temper make for a media-unfriendly mix -- as Pakistan win the cup.

Thus surrounded by awful actors, Khan bids farewell to his beloved sport, even as insufferable little kids clamber onto shoulders to get a better look at the traitor. Insert typically strained background music here, and you're cringing for both Khan and the film.

Seven years later, mercifully cutting out the tiresome Rambo-esque routine of having to persuade the self-pitying hero to return, Khan is raring to go. He hasn't been on a field since, and is eager to resolve -- as evidenced by strategic fidgeting with waiting-room bottle caps -- hockey issues.

His plan is simple: to start from the very bottom. The Indian Women's Hockey Team is an outfit so utterly neglected that its administrators aren't even actively seeking a coach. Anjan Srivastava dips a Marie biscuit in tea, raises an eyebrow, and not having anything at stake, lets Khan go for it.

So girls, then. A motley assortment of Reddys, Boses and Sharmas are picked from the length and breadth of the country, each falling into conveniently label-friendly stereotypes, but -- and here's what makes all the difference -- the tags are affectionate, the cliches run warm and friendly. And we grow to see a mostly-gangly gang of 16 indisciplined non-actresses, trying to keep up with a coach who actually takes himself seriously. And pushes them hard.

As a coach his job was not just to train his players in the game but rather get them to overcome their personal rivalry and play as a team. His challenge was to instill confidence and belief in self and in realizing that "if you want, you can do it!" He struggles but manages to bring cohesiveness among his players. Does he succeed in his effort to form Team India? Watch the movie to find out.

It's completely par for the genre-specific course -- dissent, pressure, defiance, infighting, lack of self-belief, external skepticism, and of course, ego. Again, what matters is the fluidity with which writer Jaideep Sahni has coloured inside the lines. The film's true star, Jaideep's ensured that screen-time is divided mostly evenly among the lot, yet separating a few characters for obvious star roles -- Experienced, arrogant Bindia (Shilpa Shukla); attractive, ego-driven Preeti (Segarika Ghatge); massive, Punjabi Balbir (Tanya Abrol); and pint-sized, defiant Komal (Chitrashi Rawat).

Marvellous scenes making an impact include-

- The scene wherein the women players get a standing ovation for their performance against the men.

- Introduction scene of the 16 hockey players on the ground

- The climax. The words of Kabir Khan just before the World cup final.

- The penalty shoot out scenes.. The tension, the pain was so well expressed by Khan and the girls.

Shahrukh Khan holds the film together, carrying the role with a lot of dignity and succeeds in delivering a powerful performance. The superstar portrays a role of varied emotions exceptionally well. The pain his character is going through can be felt. The bunch of 16 girls are surprisingly good.

To sum things up, Chak De India scores in all departments. Its a rare combination of a great script, brilliant direction, powerful performances and dialougues with a lot of impact. A winner all the way.
\subsection{Starring}
\begin{itemize}
	\item Shah Rukh Khan - Kabir Khan
\end{itemize}
\end{document}
